 \question[3]{Tick the $ \textbf{non-commutative} $ operation. }
{$+$ (addition) over $\mathbb{Z}$}
{$\oplus$ (exclusive-or) }
{$-$ (subtraction) over $\mathbb{Z}$}
{$\times$ (multiplication) over $\mathbb{Z}$}

 \question[2]{Tick the $ \textbf{true} $ statement. }
 {$x \in \mathbb{Z}_n$ is invertible if and only if $\mathsf{gcd}(x,n)=0$. }
 {If $x$ has order $m$ in $\mathbb{Z}_n^*$ then for any $i \in \mathbb{Z}$, $x^i \mod n = x^{i \mod m} \mod n$.}
 {In $\mathbb{Z}_n$, we have $(a \times(b \mod n)) \mod n \neq (ab) \mod n$.}
 {$\mathbb{Z}_n$ is a field with respect to the operations $(+,\times)$ for any $n$.}
 
  \question[1]{``In any finite group, the order of an element is a factor of the order of the group" is stated by}
  {the Lagrange Theorem.}
  {the Chinese Remainder Theorem.}
  {the Fermat's Little Theorem.}
  {the Fundamental Theorem of Calculus.}
  
 \question[2]{Select the non-associative operation.}
{$+$ (addition)}
{$-$ (subtraction)}
{$\times$ (multiplication)}
{$\oplus$ (exclusive-or)}


\question[1]{Tick the \emph{false} answer. In a group, the operation$ \dots $}%
 {is commutative}%
 {is associative.}%
 {has a neutral element.}%
 {associates an inverse to each value.}%
 
  \question[2]{Let $(G,+), (H,\times)$ be two groups and $f:G\to H$ be an homomorphism. For $x_1,x_2 \in G$, we have:}%
 {$f(x_1) + f(x_2)=f(x_1)\times f(x_2)$}%
 {$f(x_1 + x_2)=f(x_1)\times f(x_2)$}%
 {$f(x_1 + f(x_2))=f(x_1\times f(x_2))$}%
 {$f(x_1 \times x_2)=f(x_1)+ f(x_2)$}%
 
  \question[4]{Let $(G,+)$ be a group of order $n$. If $g$ is a generator of this group, then it has order$ \dots $}%
 {$n/2$}%
 {$\sqrt{n}$}%
 {$n-1$}%
 {$n$}%
 
\question[4]{Which of the following algorithms seen in the course has  cubic asymptotic complexity in the bit length of the operands?}
 {Multiplication}
 {Euclidean divison}
 {Extended Euclidean algorithm }
 {Exponentiation}

 \question[2]{ For $\ell$-bit numbers, the tightest complexity to find the inverse of an integer modulo another  ...}
{$\mathcal{O}(\ell)$.}
{$\mathcal{O}(\ell^2)$.}
{$\mathcal{O}(\ell^3)$.}
{$\mathcal{O}(\ell^4)$.}

\question[4]{Select \emph{incorrect} statement. }
{The Extended Euclid algorithm has polynomial complexity.}
{The Extended Euclid algorithm can be used to compute inverses.}
{The Euclid algorithm can be used to compute gcd's.}
{The Extended Euclid algorithm can be used to compute square roots.}

\question[3]{Under which condition is an element $x\in \mathbb{Z}_n$ invertible?}
 {$\mathsf{gcd}(x,\varphi (n)) = 1$.}
 {$\mathsf{gcd}(x,n-1) = 1$.}
 {$\mathsf{gcd}(x,n) = 1$.}
 {$\mathsf{gcd}(x,n) \ne 1$.}
 
 \question[3]{What is the time complexity to perfom a left-to-right multiplication of two $\ell$-bit integers?}
 {$\sqrt{\ell}$.}
 {$\ell $.}
 {$\ell ^2$.}
 {$\ell ^3$.}
 
 \question[1]{Let $n$ be a positive integer. An element $x \in \mathbb{Z}_n$ is \emph{always} invertible when $ \dots $}%
{$x$ and $n$ are coprime.}%
{$x$ and $\varphi(n)$ are coprime.}%
{$x$ is even.}%
{$n$ is prime.}%

\question[2]{Let $a$ and $b$ be two positive integers used as input of the Euclid algorithm. What is the output of this algorithm?}%
{$a^b$}%
{$\gcd(a,b)$}%
{$a \cdot b$}%
{$a/b$}%

\question[4]{Let $n$ be an integer. Which of the following is \emph{not} a group in the general case?}%
{$(\mathbf{R},+)$}%
{$(\mathbf{Q}\setminus \{0\},\times)$}%
{$(\mathbf{Z}_n,+ \pmod{n})$}%
{$(\mathbf{Z}_n,\times \pmod{n})$}%

\question[3]{Let $a$ and $b$ be two integers of at most $\ell$ digits. What is the complexity of the \emph{extended} Euclidean algorithm?}%
{$O(\ell)$}%
{$O(\ell^3)$}%
{$O(\ell^2)$}%
{$O(2^\ell)$}%

\question[2]{Let $n$ be an integer. The extended Euclidean algorithm is typically used to$ \dots $}%
{$ \dots $ perform the addition of two integers in $\mathbf{Z}_n^*$.}%
{$ \dots $ compute the inverse of an element in $\mathbf{Z}_n^*$.}%
{$ \dots $ compute the square of an element of $\mathbf{Z}_n^*$.}%
{$ \dots $ compute the order of $\mathbf{Z}_n^*$.}%

\question[4]{What is the inverse of 28 in $\mathbf{Z}_{77}$?}%
{$0$}%
{$1$}%
{$36$}%
{$49$}%

\question[3]{ Pick the \textit{true} statement regarding the Discrete Logarithm Problem.}
{It is computed efficiently if the order of the group is known.}
{It is hard to compute on any group.}
{It seems to be hard for some instances of elliptic curves.}
{It requires a fresh generator $g$ as reusing the generator always breaks the cryptosystem.}

\question[3]{Generating public parameters for the ElGamal cryptosystem of about $s$ bits requires a complexity in the order of magnitude} 
{$s^2$} 
{$s^3$} 
{$s^4$} 
{$s^5$}

 \question[4]{Tick the $ \textbf{false} $ statement. }
 {In any finite group, the order of an element divides the order of the group.}
 {The subgroups of $\mathbb{Z}$ are all of the form $n\mathbb{Z}$.}
 {$(\mathsf{GF}(2^3) ,+)$ has no generator.}
 {$(\mathbb{Z}_n,\times)$ is a group of $n$ elements.}


\question[1]{Pick the \emph{false} statement.}
{A ring is always commutative:  $ab=ba$}
{A ring is always associative:  $(ab)c=a(bc)$}
{A ring is always distributive:  $(a+b)c=ab+ac$, $(a+b)c=ac+bc$}
{A ring is always Abelian: $a+b = b+a$}

\question[1]{Select the \emph{incorrect} statement. The discrete logarithm}
 {can be solved by a polynomial algorithm.}
 {is an easy problem in the Abelian group $\mathbb{Z}_p$.}
 {is a hard problem in the multiplicative group $\mathbb{Z}_p^*$.}
 {can be solved easily on a quantum computer.}

\question[1]{Which one of these is a closed set?}
 {$\mathbb{Z}$ with the addition.}
 {$\mathbb{Z}^\star$ with the addition.}
 {$\mathbb{Z}^\star$ with the substaction.}
 {$\mathbb{Z}-\{0\}$ with the division.}

\question[2]{Let $G$ be a group generated by $g$. What is the discrete logarithm problem?}
{find $y$ such that $g^x=y$ for a given $x$.}
{find $x$ such that $g^x=y$ for a given $y$.}
{find $x,y$ such that $g^x=y$.}
{find $x,x'$ such that $g^x=g^{x'}$ and $x\ne x'$.}

\question[3]{Tick the \emph{false} assertion. The ElGamal cryptosystem$ \dots $}
{is based on the Discrete Logarithm problem.}
{produces randomized ciphertexts.}
{produces ciphertexts as long as the plaintexts.}
{encrypts messages with limited length.}

\question[2]{Which of the following elements belongs to $\mathbb{Z}_{78}^*$?}
{46}
{35}
{21}
{65}



\question[3]{What is the cardinality of the multiplicative group $\mathbb{Z}_{77}^*$?}
{70}
{66}
{60}
{76}

\question[1]{The little Fermat theorem states that for a prime $n$ and any $b\in \mathbb{Z}_n ^\star$ we have$ \dots $} 
 {$b^{n-1}\mod n = 1$.}
 {$b^{n-1}\mod n = n$.}
 {$b^{n}\mod n = 1$.}
 {$b^{n-1}\mod n = b$.}

 \question[2]{Tick the $ \textbf{false} $  statement. Let $p$ be a prime number, ...}
 {$\forall x  \in \mathbb{Z}_p$, $x \neq 0 \Longrightarrow x$ is a  generator of $\mathbb{Z}_p$.}
 {$\forall x  \in \mathbb{Z}_p^*$, $x \neq 0 \Longrightarrow x$ is  a generator of $\mathbb{Z}_p^*$.}
 {$\forall x  \in \mathbb{Z}_p^*$, $x$ is invertible.}
 {$\mathbb{Z}_p^*$ is isomorphic to  $\mathbb{Z}_{p-1}$.}


\question[2]{Tick the $ \textbf{true} $  statement.}
 {In a ring $\mathsf{R}$, $\forall a \in \mathsf{R}$, $\exists b \in \mathsf{R}$ such that $ab=1$. }
{Let $f: \mathbb{Z}_n \rightarrow \mathbb{Z}_2$ be a function such that $\forall x \in  \mathbb{Z}_n$, $f(x)=0$. Then,  $f$ is a ring homomorphism.}
{8 is an element of $\mathbb{Z}_5^3$.}
{In a general ring $\mathsf{R}$, $\forall a,b \in \mathsf{R}$,   $ab=ba$.}

  \question[3]{$\mathbb{Z}_{37}^*$ denotes ... }
 {a field.}
 {an additive group.}
 {a multiplicative group.}
 {a ring.}

 \question[1]{Which of the following is an element of  $\mathbb{Z}_{60}^*$?}
{49}
 {30}
 {26}
 {21}

 \question[4]{Tick the $ \textbf{false} $ statement.}
 {$\mathbb{Z}_n$ is a ring.}
 {$n\mathbb{Z}$ is a principal ideal of $\mathbb{Z}$.}
 {$\mathbb{Z}/n\mathbb{Z}$ is isomorphic to $\mathbb{Z}_n$.}
 {$\mathbb{Z}_n^*$ is non-commutative.}

\question[2]{Pick the \emph{correct} statement.}
{A homomorphism is defined as a bijective isomorphism.}
{An isomorphism is defined as a bijective homomorphism.}
{An isomorphism is any homomorphism $h: X\rightarrow X$.}
{A homomorphism is any non-bijective isomorphism.}

\question[3]{Choose the \emph{correct} statement.}
 {$\mathbb{Z}_n$ is a field $\Leftrightarrow$ $n$ is a composite number}
 {$\mathbb{Z}_n$ is a field $\Leftrightarrow$ $\mathbb{Z}_n^* = \mathbb{Z}_n$}
 {$\mathbb{Z}_n$ is a field $\Leftrightarrow$ $n$ is a prime}
 {$\mathbb{Z}_n$ is a field $\Leftrightarrow$ $\mathbb{Z}_n^* = \emptyset$}

 \question[3]{Tick the \emph{false} assertion. Given a ring $R$, $R^\star$ is$ \ldots $}%
 {a group.}%
 {the set of invertible elements in $R$.}%
 {$R-\{0\}$.}
 {the set of units.}%

\question[2]{Consider a group $G$ with prime order. We can deduce that}%
{all elements are generators.}%
{all elements are generators except the neutral element.}%
{half of the elements are generators.}%
{$G$ is a field.}%

\question[1]{What is the cardinality of the multiplicative group $\mathbb{Z}_{77}^*$?}%
{60}%
{66}%
{70}%
{76}%


\question[1]{Tick the $ \textbf{true} $ statement of the \textit{Fermat's Little Theorem.}}
{For all prime $n$, for all $a \in \left\{1,2, $ \dots $,n-1\right\}$  we have $a^{n-1}\equiv 1 \pmod{n}$.}
{For all $n$, for all $a \in \left\{1,2, $ \dots $,n-1\right\}$  we have $a^{n-1}\equiv 1 \pmod{n}$.}
{For all prime $n$, for all $a \in \left\{1,2, $ \dots $,n-1\right\}$  we have $a^{n-1}\not\equiv 1 \pmod{n}$.}
{For all $n$, there exists $a \in \left\{1,2, $ \dots $,n-1\right\}$  we have $a^{n-1}\equiv 1 \pmod{n}$.}

 \question[2]{Tick the non-commutative group.}
 {$\mathbb{Z}_n^*$.}
 {$S_n$, the set of all permutations over the set $\left\{1,2,$ \dots $,n \right\}$. }
 {$E_{a,b}(K)$, an elliptic curve over a field $K$. }
 {$\mathsf{GF}(2^k)^* $ .}
 
\question[2]{A passive adversary can $ \ldots $}%
{do nothing.}%
{only listen to communications.}%
{only interfere with client or server communications.}%
{only replace some communication messages by others.}% 


 
 

 
 