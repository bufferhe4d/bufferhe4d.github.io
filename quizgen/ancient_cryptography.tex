\question[4]{Tick the $ \textbf{true} $ statement.  The Caesar Cipher ...}
 {relies on Kerchkoffs's Principles. }
 {is broken due to Moore's Law.}
 {is insecure due to the Kasiski Test.}
 {is equivalent to the Vigen{\`e}re Cipher with a key of length 1. }
 
 \question[3]{Tick the $ \textbf{false} $ statement.}
{The Kasiski Test is applied to the Vigen{\`e}re Cipher.}
{Moore's Law is an empirical law.}
{The index of coincidence is useful to break the Vernam Cipher.}
{In the Vernam Cipher, the key must be used only once. }

\question[1]{Tick the $ \textbf{false} $ statement. Enigma ... }
{was  broken mainly because of design flaws in the patented documents.}
{was used by the German armies in  World War 2.}
{relies on  Kerchkoffs's Principle.}
{could be plugged into a radio transmitter.} 

  \question[4]{The $n^2$ problem ...}
 {is dealt with thanks to  Moore's Law.}
 {is a consequence of  Murphy's Law.}
 {is a direct consequence of the Kerchkoffs Principles.}
 {appears when $n$ users need to communicate to each other using a symmetric cipher.}
 
 \question[3]{The Kerckhoffs Principle states that ...}
 {the security of a cryptosystem should be based on an NP-hard problem.}
 {all ciphertexts appear with the same probability.}
 {the adversary may know the description of  a cryptosystem.}
 {the cryptosystem should be public.}
 
  \question[4]{Tick the $ \textbf{false} $ statement.  Moore's Law ...}
 {is partly a reason why some existing cryptosystems are insecure.}
 {was stated by the founder of Intel.}
 {assumes the number of transistors per CPU increases exponentially fast with time.}
 {implies that the heat generated by transistors of CPU doubles every 18 months.}
 
 \question[4]{One-time pad  ...}
 {never uses a key $K$ which is picked from a  uniform distribution.}
 {pads the message at least once before encryption.}
 {allows an efficient key management.}
 {uses an invertible group operation such as ``$\oplus$" for encryption.}
 
  \question[1]{Tick the $ \textbf{false} $ statement. The Shannon Encryption Model ...}
 {requires a black-box encryption model.}
 {assumes a known input distribution.}
 {assumes the key is independent from the message.}
 {requires the correctness property $\Pr[C_K^{-1}(C_K(X))=X]=1$.}
 
 \question[3]{A 128-bit key ...}
 {has 128 decimal digits.}
 {is too long for any practical application.}
  {provides reasonable security for at least four decades.}
  {adresses $n^2$ problem for $n=2^{64}$.}
  
  \question[4]{Exhaustive search on 128 bits is done within ...}
 {a month.}
 {a year.}
 {the lifespan of a giant turtle.}
 { more than the lifespan of our universe.}
 
  \question[2]{Tick the $ \textbf{false} $ statement regarding Kerckhoffs' principle.}
 {The system must be practically, if not mathematically, indecipherable.}
 {Besides keeping the key secret, the cryptosystem must also be kept secret.}
 {It must be applicable to telegraphic correspondence. }
 {Its key must be communicable and retainable without the help of written notes, and changeable or modifiable at the will of the correspondents.}
 
 \question[3]{Consider the Vigen{\`e}re cipher. Choose the $ \textbf{true} $ statement.}
{Kasiski test cannot be applied to the Vigen{\`e}re cipher.}
{The Vigen{\`e}re cipher constructs a basis for the modern ciphers.}
{The index of coincidence is used to determine the length of the secret key of the Vigen{\`e}re cipher.}
{The distance between the repeated patterns in the ciphertext is the exact length of the secret key.}

 \question[2]{Tick the $ \textbf{true} $ statement.}
  {The Vigen{\`e}re cipher provides perfect secrecy when the key is uniformly distributed and independent from the plaintext.}
 {Perfect secrecy implies that if $\Pr[Y=y]\neq 0$ then $\Pr[X=x|Y=y]=\Pr[X=x]$.}
 {For any key distribution, the Vernam cipher provides perfect secrecy.}
 {Message integrity means that the message is accessible only to those authorized to have access.}
 
 \question[1]{Tick the $ \textbf{false} $ statement.}
 {To have the Vernam cipher secure, it is necessary and sufficient that the key is uniformly distributed. }
 {Plaintexts and ciphertexts are statistically independent in the Vernam cipher.}
 {Using the same key twice results in the leakage of plaintexts, i.e., $X_1 \oplus X_2$.}
 {Shannon analyzed the Vernam cipher in the Shannon Encryption Model.}
 
 \question[4]{Tick the $ \textbf{false} $ statement regarding the Shannon Encryption Model.}
 {Key and Message are independent random variables with given distributions.}
 {Adversary is limited to seeing only the random variable $Y=C_K(X)$.}
 {The adversary may know the distribution of the plaintexts. }
 {It satisfies the correctness property such that $\Pr[C_K^ {-1}(C_K(X))=X]=\frac{1}{2}$.}
 
  \question[2]{Tick the $ \textbf{false} $ statement regarding the Enigma machine.}
 {It is an electro-mechanical encryption device used by German armies in World War 2.}
 {Its specifications are secret.}
 {Its secret parameters are: ordered permutations, an involution, and a number.}
 {It was patented in 1918.}
 
  \question[3]{Consider the Vernam Cipher. Let P=0010 \ 1011 and K=1110 \ 0110 and what is C=P $\oplus$ K?}
 {0011\ 0010}
 {1100\ 0111}
 {1100\ 1101}
 {1101\ 0111}
 
  \question[2]{Tick the $ \textbf{true} $ statement.}
 {In modern cryptography, we prefer to use cryptosystems which are more secure than the Vernam cipher.}
 {Having perfect secrecy requires too much cost, for instance having a huge secret key.}
 {The support of the input set X should be infinite to achieve perfect secrecy.}
 {The notion of perfect secrecy is not mathematically formalized yet.}
 
  \question[3]{Tick the $ \textbf{false} $ statement.}
{Perfect secrecy requires that the number of keys must be at least the number of plaintexts.}
{In a network of $n$ users, $\frac{n(n-1)}{2}$ secret keys are enough to build random point-to-point secure communications.}
{According to Moore's Law, the length of a key doubles every year.}
{In Moore's law, the security continuously degrades with time.} 

 \question[2]{To protect our communications, we need $$ \ldots $$}
 {the army.}
 {cryptography.}
 {to pray.}
 {a rest.}
 
 \question[4]{Choose the correct statement}
{perfect secrecy is when $H($ciphertext $|$ key$) = H($plaintext$)$}
{perfect secrecy is when $H($plaintext $|$ ciphertext$) = H($key$)$}
{perfect secrecy is when $H($ciphertext $|$ plaintext$) = H($plaintext$)$}
{perfect secrecy is when $H($plaintext $|$ ciphertext$) = H($plaintext$)$}

\question[4]{Kerckhoffs principle says}
{both the key and the encryption algorithm should be secret to achieve perfect secrecy}
{the key should be secret to achieve perfect secrecy}
{the encryption algorithm should be secret to achieve perfect secrecy}
{the security should depend only on secrecy of the key}

\question[2]{Select a correct statement}
{Morse alphabet is a cipher}
{Morse alphabet is a code}
{Morse alphabet preserves confidentiality}
{Morse alphabet preserves authenticity}

\question[4]{The entropy $H(X)$ of random variable $X$ is}
{always strictly greater than $0$}
{strictly less than $0$ if and only if $X$ is constant}
{always between $0$ and $1$; it is $0$ or $1$ only for a deterministic process}
{equal to $1$ for a flip of an unbiased coin}

\question[3]{The number of permutations on a set of $n$ elements}
{is always greater than $2^n$}
{is approximately $n(\log n - 1)$}
{can be approximated using the Stirling formula}
{is independent of the size of the set}

\question[3]{The substitution cipher with random substitution table is}
{computationally infeasible to break since number of possible keys is $26! \approx 2^{88.4}$}
{very easy to break since number of possible keys is only $26! \approx 2^{88.4}$}
{very easy to break even though number of possible keys is $26! \approx 2^{88.4}$}
{certified by NIST up to "confidential" level}

\question[1]{Index of coincidence is}
 {approximately 0.065 for English language}
 {close to 1 for any text in European language}
 {approximately 3.80 for random string}
 {close to 0 for any string}
 
 \question[2]{An involution is}
{a permutation p such that $p(p(x)) = x$ for some $x$}
{a permutation p such that $p(p(x)) = x$ for every $x$}
{any permutation which is not a transposition}
{a non injective function}

\question[2]{The Moore law}
 {implies the key size is  doubled every every 18 months to preserve confidentiality}
 {says that CPU speed doubles every 18 months}
 {has no relevance for cryptography since it only considers speed of computation}
 {states that anything that can go wrong will}
 
 \question[3]{Enigma}
  {was a predecessor of a Turing machine model - a basis of Von Neumann architecture}
  {achieves perfect security as was required due to military application}
  {follows the Kerkhoffs principle}
  {has approximately $2^{256}$ possible keys}
  
 \question[4]{Let $\oplus$ denote the exclusive-or ($\mathsf{XOR}$) operation. Tick the \emph{false} assertion.}
{$0\oplus 0 = 0$}
{$0\oplus 1 = 1$}
{$1\oplus 0 = 1$}
{$1\oplus 1 = 1$}

 \question[3]{How many bit-strings of length $n$ are there?}
{$n$}
{$n^2$}
{$2^n$}
{$\log_2 n$}

\question[4]{Which problem in communication is \emph{not} treated by cryptography?}
{confidentiality}
{integrity}
{authenthication}
{data transmission}

\question[3]{If Alice receives a message proven to be coming from Bob, we say that the message is$ \dots $}
{confidential}
{fresh}
{authenticated}
{correct}

\question[1]{The Kerckhoffs principle states that the security of a cryptosystem should rely on the secrecy of$ \dots $}
{the key only}
{the algorithm only}
{the channel only}
{the participants' identity only}

\question[4]{The Murphy Law states that if there is a single security hole in an exposed cryptosystem, then$ \dots $}
{hope for the best}
{nobody will look for it}
{nobody will find it}
{someone will ultimately find it}

\question[4]{Which one of these Ciphers is perfectly secure?}
{Hieroglyphs}
{Caesar}
{Vigen\`{e}re}
{Vernam}

\question[3]{The Shannon theorem states that perfect secrecy implies...}
{$H(K)=H(X)$}
{$H(Y)\geq H(X)$}
{$H(K)\geq H(X)$}
{$H(Y)\leq H(X)$}

 \question[3]{Using the same key twice to encrypt two different messages with Vernam Cipher leads to $ \ldots $}
 { increasing the security of the secret key.}
 { revealing the secret key.}
 { revealing information about the messages.}
 { making Shannon upset.}
 
 
 \question[2]{The drawback of perfect secrecy is that$ \ldots $}
 {it dates from 1948}
 {the key has to be as long as the message to encrypt}
 {it is not feasible on current computers}
 {it is not perfectly secret}
 
 \question[4]{Which one of the following notions is not in the
  fundamental trilogy of cryptography?}%
{authentication}%
{confidentiality}%
{integrity}%
{privacy}%

\question[1]{A simple substitution cipher can be broken $ \dots $}%
{by analysing the probability occurence of the language.}% 
{only by using a quantum computer.}% 
{by using the ENIGMA machine.}% 
{by using public-key cryptogaphy.}% 

\question[1]{How many different simple substitution ciphers do exist
  with respect to an alphabet of 26 characters?}%
{$26!$}%
{$2^{26}$}%
{$26^2$}%
{26}%

\question[3]{Which one of the following encryption method is a simple substitution cipher?}%
{Vigen\`ere cipher}%
{the Vernam cipher.}%
{the Caesar cipher.}%
{Enigma}%

\question[2]{Consider the Vigen\`ere cipher. What is the ciphertext of ``crypto'' using the secret key ``ABC''.}%
{PTOCRY}%
{CSAPUQ}%
{OTPYRC}%
{CSAPTO}%

\question[2]{Visual cryptography is a nice visual application of $ \ldots $}%
{the Vigen\`ere cipher.}%
{the Vernam cipher.}%
{the Caesar cipher.}%
{ROT13.}%

\question[3]{The Vernam cipher $ \ldots $}%
{is always secure.}%
{is secure only if we always use the same key.}%
{is secure only if we always use a different key.}%
{is always insecure.}%

\question[3]{What is the main reason for not using the Vernam cipher?}%
{The encryption step is too costly.}%
{This cipher does not guarantee the integrity.}%
{Generation of randomness and the exchange of keys are too costly.}%
{This cipher violates the Kerckhoffs principle.}%

\question[2]{The Enigma cipher $ \ldots $}%
{is as secure as the Vernam cipher.}%
{is less secure than the Vernam cipher.}%
{is more secure than the Vernam cipher.}%
{is not a cipher.}%

\question[4]{The Kerckhoffs principle says:}%
{security should not rely on the secrecy of the key.}%
{the speed of CPUs doubles every 18 months}%
{cryptosystems must be published.}%
{security should not rely on the secrecy of the cryptosystem itself.}%


 \question[1]{Which one of the following notions means that ``the information should make clear who the author of it is''?}%
{authentication}%
{steganograhy}%
{privacy}%
{confidentiality}%

  \question[2]{Visual cryptography is a nice visual application of $ \ldots $}%
{$ \ldots $ the Vigen\`ere cipher.}%
{$ \ldots $ the Vernam cipher.}%
{$ \ldots $ the Caesar cipher.}%
{$ \ldots $ ROT13.}%
 
\question[3]{Tick the \emph{false} assertion.}%
{The index of coincidence is a useful tool to break the Vigen\`ere cipher.}%
{The index of coincidence is invariant under substitution.}%
{The Kasiski test makes use of the index of coincidence.}%
{The Kasiski test is a useful tool to break the Vigen\`ere cipher.}%

\question[4]{The Enigma cipher $ \ldots $}%
{$ \ldots $ never existed.}%
{$ \ldots $ was invented by Kasiski.}%
{$ \ldots $ does not respect the Kerckhoffs principle.}%
{$ \ldots $ is less secure than the Vernam cipher.}%
 
 
\question[3]{The composition of a simple substitution cipher with itself corresponds to $ \ldots $ }%
{$ \ldots $  a Vigen\`ere cipher with a key of length 2.}%
{$ \ldots $ the identity.}%
{$ \ldots $ another simple substitution cipher.}%
{$ \ldots $ the Caesar cipher.}% 


\question[2]{The Kasiski Test is useful to break$ \dots $}%
{$ \dots $ the simple substitution cipher.}%
{$ \dots $ the Vigen\`ere Cipher.}%
{$ \dots $ a Turing machine.}%
{$ \dots $ Enigma.}% 

\question[4]{According to the Kerckhoffs Principle:}%
{The internal design of a cryptosystem should be public.}%
{The internal design of a cryptosystem should \emph{not} be public.}%
{If there is a single security hole in a cryptosystem, somebody will discover it.}%
{The security of the cryptosystem should \emph{not} rely on the secrecy of the cryptosystem itself.}% 

\question[1]{The Vernam cipher$ \dots $}%
{$ \dots $ is perfectly secure (if used in a proper way).}%
{$ \dots $ can be broken using an index of coincidence.}%
{$ \dots $ is always less secure than DES.}%
{$ \dots $ has a security which depends on the Moore law.}% 

\question[4]{Visual Cryptography is a good illustration of$ \dots $}%
{$ \dots $ DES.}%
{$ \dots $ Enigma.}%
{$ \dots $ the Simple Substitution cipher.}%
{$ \dots $ the Vernam cipher.}% 

\question[1]{Tick the \emph{true} assertion among the followings:}%
{Visual cryptography is perfectly secure (at an unreasonable cost).}%
{The Vernam cipher was invented by Kerckoff.}%
{Just like coding theory, cryptography usually faces random noise.}%
{Enigma has never been broken.}% 

\question[2]{Which of these components was not part of the Enigma machine?}%
{A reflector}%
{A pseudo-random number generator}%
{A Rotor}%
{A plugboard with a wire connection}% 

\question[4]{Which of these plaintexts can be the result of the decryption of $ \texttt{SERGEV} $ using a simple subsitution cipher?}%
{$ \texttt{VERNAM} $}%
{$ \texttt{ENIGMA} $}%
{$ \texttt{TURING} $}%
{$ \texttt{CAESAR} $}% 

\question[3]{What is the result of the encryption of $ \texttt{CRYPTO} $ using ROT13?}%
{$ \texttt{SERGEV} $}%
{$ \texttt{FUBSWR} $}%
{$ \texttt{PELCGB} $}%
{$ \texttt{YNULPK} $}% 

\question[2]{ Ensuring the information integrity means that$ \dots $}%
{$ \dots $ the information should not leak to any unexpected party.}%
{$ \dots $ the information must be protected against any malicious modification.}%
{$ \dots $ the information should make clear who the author of it is.}%
{$ \dots $ DES is secure.}% 
