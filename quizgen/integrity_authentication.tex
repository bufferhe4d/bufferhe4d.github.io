\question[2]{Tick the <strong>false</strong> assertion.}
{For MACs, in the chosen message attack model, the adversary can force the sender to authenticate a selected message.}
{For MACs, the goal of an adversary is to decrypt a message.}
{For MACs, in the known message attack model, the adversary can only read authenticated messages in transit.}
{For some MACs, the adversary can forge authenticated messages without recovering the secret key.}
	
\question[1]{Tick the <strong>false</strong> assertion.}
 {$\epsilon$-universal hash functions can be used universally, i.e., in any cryptographic application.}
 {$\epsilon$-universal hash functions are families of functions.}
 {An $\epsilon$-XOR-universal hash function $h_k$, for given $x,y$, $x \neq y$ satisfies that for any $a$ $\Pr[h_k(x)\oplus h_k(y) = a] \leq \epsilon$ over a random $k$.}
 {$\epsilon$-universal hash functions are used in WC-MAC.}
 
\question[2]{Birthday attacks $ \dots $}
{are used to break Google Calendars.}
{can be used to find collisions in hash functions.}
{are equivalent to exhaustive search.}
{imply that a majority of people is born in Spring.}

\question[3]{Tick the <strong>correct</strong> assertion.}
{The only way of finding a collision in a hash function while using (almost) no memory is exhaustive search.}
{Floyd's cycle finding algorithm returns the closest bike shop.}
{Exhaustive search is always a concern in cryptographic schemes.}
{$\mathbb{Z}_p^*$ is a field.}

 \question[1]{Tick the <strong>false</strong>  statement.}
 {If the connection polynomial has a prime degree, then the LFSR has maximal period.}
 {GSM A5/1 is a hardware dedicated lightweight stream cipher.}
 {A nonce is a number which is used only once.}
 {The CCM mode provides authenticity.}

 
 \question[1]{Which MAC construction is NOT based on the CBC mode?}
{HMAC}
{EMAC}
{OMAC}
{ISO/IEC 9797}

\question[3]{Given a message $x$, find a different message $x'$ such that  $h(x)=h(x')$ is the definition of ... }
{First preimage attack.}
{Collision attack.}
{Second preimage attack.}
{Known message attack.}


\question[2]{The Generic Collision Search Algorithm is based on ... }
{the Dictionary Attack.}
{ Birthday Paradox.}
{the simulation of quantum state on a standard machine.}
{Murphy's Law.}

\question[3]{Finding collisions on a set of N elements  ... }
{requires the storage of size $ \Omega(N).$}
{requires time $ O({N}^{\frac{1}{3}}).$}
{can be done with the storage of size  $O(1).$}
{is doable for $N=2^{256}$.}

\question[2]{The Time-Memory Tradeoff Attack ...  }
{is useful for finding a preimage within complexity $O\big(\big({\frac{2}{3}}\big)^N\big).$}
{is useful for finding a preimage within complexity $O(N^{\frac{2}{3}}).$}
{is a dedicated method which works only on SHA1.}
{can be combined with birthday paradox to find the order of the group in RSA efficiently.}


\question[2]{Tick the $ <strong>true</strong> $ statement.}
{In order to have $k$-bit security of a hash function of $h$-bit digest size against the collision attack, we need $h=k$.}
{The probability of finding a collision in AES for a fixed key is 0.}
{In the Dictionary Attack, using a smaller dictionary increases the probability of success.}
{The average complexity of finding a preimage onto a domain of size $N$ is $O(N^{\frac{1}{2}})$.}

\question[4]{Select the <strong>incorrect</strong> statement: hash functions can be used to construct}
{commitment schemes}
{key derivation functions}
{message authentication codes}
{public key cryptosystems}

\question[2]{Which one is <strong>not</strong> a security requirement on hash functions}
{onewayness}
{bijectivity}
{pseudo-randomness}
{collision resistance}

\question[3]{The Davies-Meyer construction is}
{a method of iterating a hash function to obtain a compression function.}
{a method of iterating a compression function to obtain a hash function.}
{a method of constructing a compression function from a block cipher.}
{a method of constructing a block cipher function from a hash function.}

\question[4]{MD5 is}
 {a secure block cipher}
 {a broken block cipher}
 {a secure hash function}
 {a broken hash function}
 
 

\question[1]{Choose the <strong>incorrect</strong> statement.}
 {The key is always sent alongside the commitment.}
 {Statisticaly hiding property is more desirable than computationally hiding.}
 {A commitment scheme can be perfectly hiding.}
 {A commitment scheme can be non-interactive.}
 
 \question[3]{MAC stands for}
 {Message Access Control}
 {Mutually Authenticated Code}
 {Message Authentication Code}
 {the name of the Scotish inventor}
 
\question[4]{Select the <strong>incorrect</strong> statement}
 {RC4, A5/1, E0 are stream ciphers}
 {MD4, MD5, SHA0, SHA1 are hash functions}
 {DES, AES are block ciphers}
 {ECB, KDF, PRF are commitment schemes}

\question[1]{Select <strong>incorrect</strong> statement. Password recovery technique from hash function}
{is based on looking for preimage.}
{is usually a dictionary attack.}
{is more expensive if we hash password with a random salt.}
{is prevented by having long random looking passwords.}

\question[4]{Select <strong>incorrect</strong> statement. The birthday paradox}
{implies that in class of $23$ students we have two student with same birthday with approximately $50\%$ probability.}
{can be used to find collisions in hash function.}
{implies that in a list of $\Theta\sqrt{N}$ random numbers from $\mathbb{Z}_N$ we have at least one number twice with probability $1- e^{-{\Theta^2\over 2}}$.}
{implies that majority of people is born at full moon.}

\question[3]{Select <strong>incorrect</strong> statement. Brithday paradox}
{is a brute force technique.}
{can be implemented with constant memory using Rho ($\rho$) method.}
{is used to recover the secret key of AES in $2^{64}$ computations.}
{can be implemented using a table of size $\Theta\sqrt{N}$}

\question[3]{What is the strongest notion for the binding property of a commitment scheme?}
{no binding.}
{statistically binding.}
{perfectly binding.}
{computationally binding.}

\question[4]{Which one of these algorithms is <strong>not</strong> defined for a MAC?}
{Generator}
{MAC}
{Check}
{Encrypt}

\question[3]{Which one of these is <strong>not</strong> a MAC construction?}
{HMAC}
{OMAC}
{MD5}
{ISO/IEC 9797}

\question[1]{A MAC forgery is$ \dots $}
{a valid pair $(X,c)$ produced by the adversary.}
{a valid pair $(X,c)$ produced by the MAC issuer.}
{a valid pair $(X,c)$ produced by the MAC verifier.}
{a key recovery attack.}

\question[4]{The collision resistance property of a hash function $H$ means that it is infeasible to$ \dots $}
{find $Y$ such that $H(X)=Y$ for a given $X$.}
{find $X$ such that $H(X)=Y$ for a given $Y$.}
{find $X'$ such that $H(X')=H(X)$ and $X\ne X'$ for a given $X$.}
{find $X,X'$ such that $H(X)=H(X')$ and $X\ne X'$.}

\question[1]{A Davies-Meyer Scheme <strong>not</strong> is used in$ \dots $}
{the Merkle-Damgard's extension.}
{AES.}
{MD5.}
{SHA-1.}

\question[4]{If we pick independent random numbers in $\{1, 2, $ \dots $, N\}$ with uniform distribution, $\theta \sqrt{N}$ times, we get at least one number twice with probability$ \dots $}
{$e^{\theta ^2}$}
{$1-e^{\theta ^2}$}
{$e^{-\theta ^2 /2}$}
{$1-e^{-\theta ^2 /2}$}

\question[2]{Due to the birthday paradox, a collision search in a hash function with $n$-bit output has complexity$ \dots $}
{$2^{\sqrt{n}}$}
{$\sqrt{2^n}$}
{$2^n$}
{$2^{n-1}$}

\question[3]{You are given the task of choosing the parameters of a hash function. What value of the output will you recommend in order to be minimal and secure against second preimage attacks?}
{40 bits}
{80 bits}
{160 bits}
{320 bits}

\question[2]{Tick the <strong>false</strong> assertion. A hash function can have the following roles:}%
{Domain extender.}%
{Block cipher.}%
{Commitment.}%
{Pseudorandom generator.}%

\question[3]{Let $H$ be a hash function. Collision resistance means that $ \dots $}%
{given $y$, it is hard to find $x$ such that $H(x)=y$}%
{given $x$, it is hard to find $y$ such that $H(x)=y$}%
{it is hard to find $x_1$ and $x_2\neq x_1$ such that $H(x_1)=H(x_2)$}%
{given $x_1$, it is hard to find $x_2\neq x_1$ such that $H(x_1)=H(x_2)$}%

\question[2]{The Davis-Meyer scheme is used to $ \ldots $}%
{build a signature using a hash function.}%
{build a compression function using an encryption scheme.}%
{build a public-key cryptosystem using a block cipher.}%
{build a block cipher using a stream cipher.}%

\question[1]{Let $H$ be a hash function based on the Merkle-Damg{\aa}rd construction. The Merkle-Damg{\aa}rd theorem says that $ \dots $}%
{$ \dots $ $H$ is collision-resistant when the compression function is collision-resistant.}%
{$ \dots $ the compression function is collision-resistant when $H$ is collision-resistant.}%
{$ \dots $ $H$ is collision-resistant.}%
{$ \dots $ $H$ is not collision-resistant.}%

\question[2]{Let $H:\{0,1\}^* \rightarrow \{0,1\}^n$ be a hash function and $x_1,x_2\in\{0,1\}^{2n}$ two random different messages. In cryptography, we usually assume that the probability of collision, i.e. $\Pr[H(x_1)=H(x_2)]$, is close to $ \ldots $}%
{$2^{-\frac{n}{2}}$.}%
{$2^{-n}$.}%
{$2^{-2n}$.}%
{$0$.}%

 \question[4]{Tick the <strong>minimal</strong> assumption on the required channel to exchange the key of a Message Authentication Code (MAC):}%
 {nothing.}%
 {authentication and integrity only.}%
 {confidentiality only.}%
 {authentication, integrity, and confidentiality.}%
 
 \question[3]{In a strong adversarial model against a Message Authentication Code (MAC), the adversary can $ \ldots $}%
{get samples of authenticated messages.}%
{get samples of encrypted messages.}%
{request the authentication of several messages.}%
{request the encryption of several messages.}%

\question[1]{The CBCMAC is $ \ldots $}%
{insecure.}%
{insecure only when using AES.}%
{secure only when using AES.}%
{secure.}%

\question[2]{Consider a MAC defined by $\mathsf{MAC}: \{0,1\}^* \times \{0,1\}^k \mapsto \{0,1\}^n$. The complexity of a generic key recovery attacks against $\mathsf{MAC}$ is $ \ldots $}%
{$2^{k/2}$}%
{$2^k$}%
{$2^{n/2}$}%
{$2^n$}%

\question[4]{Using a block cipher, we can build $ \ldots $}%
{only hash functions.}%
{only MACs.}%
{only hash functions and MACs.}%
{hash functions, MACs, and stream ciphers.}%

\question[4]{A hash function $h$ is collision-resistant if$ \dots $}
{$ \dots $ given $y$, it is hard to find $x$ such that $h(x)=y$}
{$ \dots $ given $x$, it is hard to find $y$ such that $h(x)=y$}
{$ \dots $ given $x$, it is hard to find $x' \ne x$ such that $h(x)=h(x')$}
{$ \dots $ it is hard to find $x,x'$ such that $x \ne x'$ and $h(x) = h(x')$}

 \question[3]{Which of the following algorithms is <strong>not</strong> a hash function?}
 {SHA-1}
 {MD5}
 {RC4} 
 {MD4}
 
 
 \question[3]{Let $h$ be a cryptographic hash function based on the Merkle-Damg{\aa}rd scheme. The Merkle-Damg{\aa}rd Theorem states that$ \dots $}
 {$ \dots $ $h$ is collision-resistant.}
 {$ \dots $ $h$ is resistant to a first preimage attack.}
 {$ \dots $ if the compression function is collision-resistant, then $h$ is collision-resistant.}
 {$ \dots $ if $h$ is collision-resistant, then the compression function is collision-resistant.}
 
\question[2]{Given a generic hash function $h$, which of the following attacks has the lowest complexity?}
{first preimage attack}
{collision search}
{second preimage attack}
{multi-collision search}

\question[1]{You are given a hash function $h$ based on the Merkle-Damg{\aa}rd scheme. Which of the following attacks is the most difficult, <strong>a priori</strong>?}
 {first preimage attack}
 {collision search}
 {second preimage attack}
 {collision search on the compression function}

 \question[1]{Which of the following cryptographic primitives does <strong>not</strong> use a key?}
  {A hash function}
  {A block cipher}
  {A Message Authentication Code (MAC)}
  {A stream cipher}
  
  
 \question[4]{Tick the <strong>false</strong> assertion.}
 {Because of the birthday paradox, one needs approximately $2^{n/2}$ messages to find a collision on a generic $n$ bit hash function.}
 {If one picks $\sqrt{N}$ random numbers in $\{1,2,$ \dots $,N\}$, the probability of obtaining at least one number twice is approximately $1-e^{-1/2}$.}
 {MAC are <strong>not</strong> concerned by the Birthday Paradox as much as hash functions are.}
 {HMAC is based on the CBC mode of operation of block ciphers.}
 
 \question[4]{Which of following algorithms produces digest of 160 bits?}
 {MD4}
 {MD5} 
 {RC4}
 {SHA-1}
 
 \question[4]{What should the minimal length of the output of a hash function be to provide security against <strong>collision attacks</strong> of $2^{256}?$}
{$2^{256}$ bits.}
{$2^{512}$ bits.}
{$256$ bits.}
{$512$ bits.}
	
\question[3]{What should the minimal length of the output of a hash function be to provide security against <strong>preimage attacks</strong> of $2^{256}?$}
{$2^{256}$ bits.}
{$2^{512}$ bits.}
{$256$ bits.}
{$512$ bits.}

\question[4]{Tick the <strong>correct</strong> assertion.}
{DES uses 256-bit keys.}
{AES uses 80-bit keys.}
{MD5 has a 160-bit digest.} 
{SHA-1 has a 160-bit digest.}
	
\question[1]{Moore's Law ...} 
{is an empirical law.}
{says that the cost of computers doubles every 18 months.}
{will allow to break AES in 2015.}
{is a main reason for discarding MD5 hash function.}