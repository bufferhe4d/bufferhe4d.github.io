\question[3]{Tick the $ \textbf{true} $ assertion.}
{In a known-message attack against a MAC, the adversary is more powerful than in a chosen-message attack.}
{A MAC can never provide authentication.}
{A MAC may protect integrity.}
{The checking algorithm inside a MAC only requires a public key."}

\question[2]{Tick the $ \textbf{false} $ assertion.}
{Traditionally, a commitment scheme is considered secure if it is binding and hiding.}
{A commitment scheme is binding if and only if it is hiding.}
{Commitment schemes can be obtained using hash functions.}
{One may commit to a message of an arbitrary length."}

\question[3]{Tick the $ \textbf{true} $ assertion.}
{Hash functions cannot be used to produce pseudo-random generators.}
{A collision attack mounted on a hash function is a special case of first pre-image attacks.}
{In applications, we can use hashes to enlarge the domain of the exchanges.}
{Hash functions map a message of a fixed length to a message of an arbitrary length that looks random."}

\question[4]{Tick the $ \textbf{true} $ assertion. Let $n >1 $ be a composite integer, the product of two primes. Then,}
{$\phi(n)$ divides $\lambda(n)$.}
{$\lambda(n)$ divides the order of any element $a$ in $\mathbb{Z}_n$.}
{$\mathbb{Z}^{*}_n$ with the multiplication is a cyclic group.}
{$a^{\lambda(n)} \mod n=1$, for all $a \in \mathbb{Z}^{*}_n$."}

\question[1]{Tick the $ \textbf{false} $ assertion.}
{The square-and-multiply algorithm cannot be used when multiplying a constant with a point on elliptic curve.}
{The extended Euclid algorithm can be used to find the inverse of a number modulo an integer $n$.}
{Generating the modulus of the RSA takes $O(\ell^4)$, where $\ell$ is the size of its factors.}
{Normally, multiplying two integers of size $\ell$ takes $O(\ell^2)$."}

\question[2]{Tick the $ \textbf{true} $ assertion.}
{The discrete logarithm problem is used in cryptography because it is a hard problem in all groups.}
{The group of quadratic residues modulo a prime $p$ has the order $\frac{p-1}{2}$.}
{The NFS algorithm is an algorithm for inverting integers modulo a prime $p$.}
{The Chinese Remainder theorem can always be used to find the solution in $\mathbb{Z}_{m_{1} \times m_{2} \times \ldots \times m_{n}}$ of a system with equations in $\mathbb{Z}_{m_{1}}, \mathbb{Z}_{m_{2}}, \ldots, \mathbb{Z}_{m_{n}}$."}

\question[1]{Tick the $ \textbf{false} $ assertion.}
{A NP-hard problem cannot be reduced to another NP-hard problem.}
{A hash produces messages of length $n$. Then, the finding of a collision is in $O(\sqrt{2^n})$.}
{A probabilistic Turing machine is non-deterministic Turing machine.}
{If a problem $P_1$ can be Turing-reduced to a problem $P_2$, then $P_1$ is not harder than~$P_2$."}

\question[1]{Tick the $ \textbf{true} $ assertion. Let $X$ be a random variable defined by the visible face showing up when throwing a dice. Its expected value $E(X)$ is:}
{3.5}
{3}
{1}
{4"}

\question[2]{Tick the $ \textbf{true} $ assertion. Let $X,Y$ be two random variables over the same probability space. Then,}
{$X$ is always independent from $Y$.}
{$E(XY)=E(X)\times E(Y)$, if $X$ and $Y$ are independent.}
{$\Pr[X = x \, \text{and} \, Y = y ] = \Pr[X = x ] \times \Pr[Y = y]$.}
{$X+Y$ does not make sense."}

\question[3]{Standard encryption threats do not include:}
{Known-plaintext attacks.}
{Chosen-plaintext attacks.}
{Universal forgeries.}
{Key-recovery attacks."}

\question[1]{Tick the $ \textbf{false} $ assertion.}
{In a known message attack against MACs, the attacker typically needs to request the authentication of several messages.}
{A MAC provides authentication.}
{A MAC may provide integrity.}
{MACs can be designed using block ciphers, stream ciphers or hash functions."}

\question[4]{Tick the $ \textbf{false} $ assertion.}
{Commitment schemes should be binding.}
{Commitment schemes should be hiding.}
{Computationally hard problems are used to ensure the binding property.}
{The binding property forces the receiver to open the commitment."}

\question[3]{Tick the $ \textbf{true} $ assertion. Let $a,b,c \in \mathbb{Z}$, with $a$ and $b$ not both zero. The equation $ax+by=c$ can be solved in $\mathbb{Z} \times \mathbb{Z}$}
{if and only if $c$ is not prime.}
{if and only if $a$ and $b$ are coprime.}
{using the extended Euclid algorithm.}
{using square root with factorization."}

\question[3]{Tick the $ \textbf{false} $ assertion.}
{$\langle \mathbb{Z}_n, + \rangle$ is a cyclic group with generator $1$.}
{$\langle \mathbb{Z}^{*}_4, \cdot \rangle$ is a cyclic group.}
{The group order is always greater than the order of an element in the group.}
{The group exponent divides the group order."}

\question[1]{Tick the $ \textbf{true} $ assertion. Let $X$ be a random variable that is equal to zero with probability 1/2 and to 1 with probability 1/2. Since the variance of $X$ is $V(X)=E((X-E(X))^2)= E(X^2)-(E(X))^2$, here $V(X)$ is:}
{1/4}
{1/2}
{1/3}
{1/8"}

\question[1]{Tick the $ \textbf{false} $ assertion.}
{A MAC provides confidentiality.}
{A MAC provides authentication.}
{A MAC provides integrity.}
{A MAC checking algorithm can be achieved by simply recomputing the MAC."}

\question[2]{Commitments should \ldots}
{resist forgery.}
{be binding and hiding.}
{allow a third party to open commitments.}
{resist preimage attacks."}

\question[3]{Tick the $ \textbf{true} $ assertion.}
{The achieved security strength is inversely proportional to the adversary power.}
{The achieved security strength is inversely proportional to the adversary capabilities.}
{Stronger security is achieved when resisting to adversaries with high capabilities.}
{Protection against low power adversaries provide stronger security."}

\question[3]{What is the Lagrange property? ($a\mid b$ means that $a$ is a factor of $b$)}
{Group order $\mid$ group exponent $\mid$ element order.}
{Group order $\mid$ element order $\mid$ group exponent.}
{Element order $\mid$ group exponent $\mid$ group order.}
{Element order $\mid$ group order $\mid$ group exponent."}

\question[1]{Tick the $ \textbf{true} $ assertion.}
{An ideal is a subgroup of a Ring.}
{All elements of an ideal are invertible.}
{A principal ring is a field.}
{All rings have cardinality of the form $p^n$ where $p$ is prime."}

\question[4]{Tick the $ \textbf{true} $ assertion. $x\in \mathbf{Z}_{n}$ is invertible iff \ldots}
{$\varphi(n)= n-1$.}
{$x$ is prime.}
{$x$ is not prime.}
{$gcd(x,n) = 1$."}

\question[3]{Tick the $ \textbf{true} $ assertion. Assume that $p$ is prime.}
{$QR(p)$ is of order $\frac{p-1}{4}$}
{$\mathbf{Z}_{p}^*$ has only one generator.}
{$\mathbf{Z}_{p}^*$ has $\varphi(\varphi(p))$ generators.}
{All elements of $\mathbf{Z}_{p}$ are invertible."}

\question[4]{Tick the $ \textbf{true} $ assertion.}
{It is asymptotically harder to do a collision than to do a preimage attack.}
{The probability that a random number is prime increases whith the increase of size length.}
{If $f(n)\in O(g(n))$ then $f(n)\in \Theta(g(n))$.}
{If $f(n)\in \Theta(g(n))$ then $f(n)\in O(g(n))$."}

\question[4]{What is the desired functionality of Symetric Encryption for a key $K$ and a message $M$?}
{Dec$_{M}($Enc$_{M}(K))=K$}
{Enc$_{M}($Dec$_{M}(K))=K$}
{Enc$_{K}($Dec$_{K}(M))=M$}
{Dec$_{K}($Enc$_{K}(M))=M$"}

\question[2]{What is the name of the encryption threat that corresponds to $ \textbf{force the sender to encrypt some messages selected by the adversary} $?}
{Chosen Ciphertext Attack}
{Chosen Plaintext Attack}
{Known Ciphertext Attack}
{Known Plaintext Attack"}

\question[3]{Tick the $ \textbf{true} $ assertion. MAC is \ldots}
{\ldots a computer.}
{\ldots the name of a dish with chili.}
{\ldots a Message Authentication Code.}
{\ldots the encryption of KEY with the Ceasar cipher."}

\question[1]{Tick the $ \textbf{true} $ assertion. A Universal Forgery is \ldots}
{\ldots a forgery where the adversary is able to forge a valid MAC/signature for an arbitrary message.}
{\ldots a forgery where the adversary is able to forge a valid MAC/signature for a new message.}
{\ldots a forgery where the adversary has to recover the secret key.}
{\ldots a forgery where the adversary plays chess."}

\question[3]{Tick the $ \textbf{true} $ assertion. A first preimage attack on a hash function H is \ldots}
{\ldots given $x$ find $y$ such that $H(x)=y$}
{\ldots given $x$ find $x'\neq x$ such that $H(x)=H(x')$}
{\ldots given $y$ find $x$ such that $H(x)=y$}
{\ldots find $x$ and $x'$ such that $x'\neq x$ and $H(x)=H(x')$"}

\question[3]{Tick the $ \textbf{false} $ assertion.}
{A commitment Commit$(X,r)$ is deterministic when $r$ is fixed.}
{In a key agreement protocol, algorithms run at different sites with no common secret lead to the same output $K$.}
{A public-key cryptosystem is composed by the following three algorithms: PRNG, encryption and decryption}
{A digital signature scheme is composed by the following three algorithms: key generator, signature and verification"}

\question[1]{Tick the $ \textbf{false} $ assertion.}
{$\mathbf{Z}_{15}^*$ is cyclic.}
{$\mathbf{Z}_4^*$ is cyclic.}
{$\mathbf{Z}_2^*$ is cyclic.}
{$\mathbf{Z}_{23}^*$ is cyclic."}

\question[4]{Tick the $ \textbf{false} $ assertion.}
{$x^{\varphi(n)}\mod n = 1,\ \forall x \in \mathbf{Z}_n^*$}
{$x^{\frac{p-1}{2}}\mod p = 1 \Leftrightarrow x\in QR(p)$}
{$\varphi(p_1^{\alpha_1}\ldots p_r^{\alpha_r})=p_1^{\alpha_1}\ldots p_r^{\alpha_r}(1-\frac{1}{p_1})\ldots(1-\frac{1}{p_r})$}
{$\lambda(p_1^{\alpha_1}\ldots p_r^{\alpha_r})=(p_1-1)p_1^{\alpha_1-1}\ldots(p_r-1)p_r^{\alpha_r-1}$"}

\question[2]{What is the complexity of prime number generation for a prime of length $\ell$?}
{$\mathbf{O}\left(\frac{1}{\ell^4}\right)$}
{$\mathbf{O}(\ell^4)$}
{$\Theta\left(\frac{1}{\ell^4}\right)$}
{$\Theta(\ell^4)$"}

\question[4]{Which one of these is not believed to be a hard problem?}
{$\mathbf{NP}$-hard problems.}
{Factoring problem.}
{Discrete logarithm problem.}
{Computing the CRT."}

\question[2]{Which one of these ciphers does achieve perfect secrecy?}
{RSA}
{Vernam}
{DES}
{FOX"}

\question[2]{Let $X$ be a plaintext and $Y$ its ciphertext. Which statement is $ \textbf{not} $ equivalent to the others?}
{the encyption scheme provides perfect secrecy}
{only a quantum computer can retrieve $X$ given $Y$}
{$X$ and $Y$ are statistically independent random variables}
{the conditionnal entropy of $X$ given $Y$ is equal to the entropy of $X$"}

\question[3]{Tick the $ \textbf{false} $ assertion. For a Vernam cipher...}
{SUPERMAN can be the result of the encryption of the plaintext ENCRYPT}
{CRYPTO can be used as a key to encrypt the plaintext PLAIN}
{SERGE can be the ciphertext corresponding to the plaintext VAUDENAY}
{The key IAMAKEY can be used to encrypt any message of size up to 7 characters"}

\question[3]{The Shannon theorem states that perfect secrecy implies...}
{$H(K)=H(X)$}
{$H(K)>H(X)$}
{$H(K)\geq H(X)$}
{$H(K)< H(X)$"}

\question[3]{Tick the $ \textbf{false} $ assertion. Vernam is perfectly secure when \ldots}
{\ldots the key is at least as long as the message}
{\ldots a key is never used twice}
{\ldots the key is a random prime number}
{\ldots the key source is truly random"}

\question[3]{Using the same key twice to encrypt two different messages with Vernam Cipher leads to \ldots}
{\ldots increasing the security of the secret key.}
{\ldots revealing the secret key.}
{\ldots revealing information about the messages.}
{\ldots nothing."}

\question[2]{Let $X$, $Y$, and $K$ be respectively the plaintext, ciphertext, and key distributions. $H$ denotes the Shannon entropy. Considering that the cipher achieves $ \textbf{perfect secrecy} $, tick the $ \textbf{false} $ assertion:}
{$X$ and $Y$ are statistically independent}
{$H(X,Y)=H(X)$}
{VAUDENAY can be the result of the encryption of ALPACINO using the Vernam cipher.}
{$H(X|Y)=H(X)$"}

\question[1]{Let $X$, $Y$, and $K$ be respectively the plaintext, ciphertext, and key distributions. $H$ denotes the Shannon entropy. The consequence of perfect secrecy is \dots}
{$H(K) \geq H(X)$}
{$H(K) \leq H(X)$}
{$H(K,X) \leq H(X)$}
{$H(Y) \leq H(X)$"}

\question[1]{Which adversarial model makes sense for PRNGs.}
{Indistinguishability.}
{Known message attack.}
{Existential forgery.}
{Perfectly binding."}

\question[4]{For any random variable $X$ with $E[X] = 0$, $E[2X^2 + 3X] = \dots$}
{$2$.}
{$4\textsf{Var}(X)$.}
{$0$.}
{$2\textsf{Var}(X)$."}

\question[2]{Tick the $ \textbf{correct} $ assertion.}
{Factoring is known to be $\mathcal{NP}$-hard.}
{A membership problem in $\mathcal{P}$ is necessarily in $\mathcal{NP}$.}
{Suppose I can solve a specific problem in $\mathcal{NP}$ in polynomial time, then I can solve any other problem in $\mathcal{NP}$ in polynomial time.}
{Computing discrete logarithms is known to be $\mathcal{NP}$-hard."}

\question[4]{Tick the $ \textbf{incorrect} $ assertion when $x\rightarrow+\infty$.}
{$x^2+5x+2 = O(x^3)$.}
{$x^n = O(n^x)$ for any constant $n > 1$.}
{$x\log(x) = O(x^2)$.}
{$1/x = O(1/x^2)$"}

\question[1]{Tick the $ \textbf{incorrect} $ statement for $ \textbf{independent} $ random variables $X$ and $Y$.}
{$Var(XY) = Var(X)Var(Y)$.}
{$E[XY] = E[X]E[Y]$.}
{$\Pr[X=x\text{ and } Y = y] = \Pr[X=x]\Pr[Y=y]$.}
{$E[X+Y] = E[X] + E[Y]$."}

\question[3]{I want to send a value to Bob without him knowing which value I sent and such that I cannot change my mind later when I reveal it in clear. I should use \dots}
{a stream cipher.}
{a PRNG.}
{a commitment scheme.}
{a digital signature."}

\question[2]{For $K$ a field, $a,b\in K$ with $4a^3+27b^2 \neq 0$, $E_{a,b}(K)$ is}
{a field.}
{a group.}
{a ring.}
{a ciphertext."}

\question[3]{Let $p>2$ be a prime. Then \dots}
{for any $x \in \mathbb{Z}_p^*$, we have $x^p \bmod{p} = 1$.}
{the set of quadratic residues modulo $p$ form a field.}
{the set of quadratic residues modulo $p$ is of order $(p-1)/2$.}
{$\phi(p^2) = (p-1)^2$."}

\question[1]{Tick the $ \textbf{incorrect} $ assertion.}
{By the CRT, $f: \mathbb{Z}_9 \rightarrow \mathbb{Z}_3\times \mathbb{Z}_3$ defined by $f(x) = (x\bmod{3}, x\bmod{3})$ is a ring isomorphism.}
{By the CRT, if $\gcd(m,n) = 1$, then $\phi(mn) = \phi(m)\phi(n)$.}
{By the CRT, there is a ring isomorphism between $\mathbb{Z}_{26}$ and $\mathbb{Z}_{13}\times \mathbb{Z}_2$.}
{One can use the CRT to simplify the search of square roots in $\mathbb{Z}_{\prod_i p_i}$, for distinct primes $p_i$."}

\question[2]{Tick the $ \textbf{correct} $ assertion.}
{If I can find a collision in a hash function, I can easily mount a first preimage attack.}
{A hash function can be used to increase the size of the domain of a primitive.}
{If I can find a collision in a hash function, I can easily mount a second preimage attack.}
{It is possible to design a hash function that has no collisions at all."}

\question[4]{For a $n$-bit block cipher with $k$-bit key, given a plaintext-ciphertext pair, a key exhaustive search has an average number of trials of \dots}
{$2^n$}
{$2^k$}
{$\frac{2^n+1}{2}$}
{$\frac{2^k+1}{2}$"}

\question[1]{Which of the following attacks needs no precomputation.}
{Exhaustive search}
{Dictionary attack}
{Time-Memory tradeoff}
{Time-Memory tradeoff with a rainbow table"}

\question[3]{The worst case complexity of an exaustive search (with memory) against DES is\dots}
{$1$}
{$\frac{2^{64}}{2}$}
{$2^{56}$}
{$2^{64}$"}

\question[2]{DES uses a key of size\dots}
{$48$ bits}
{$56$ bits}
{$64$ bits}
{$128$ bits"}

\question[1]{A Feistel scheme is used in\dots}
{DES}
{AES}
{FOX}
{CS-Cipher"}

\question[4]{A collision-resistant hash function $h$ is a hash function where it is $ \textbf{infeasible} $ to find\dots}
{a digest $d$ such that, given a message $m$, $h(m)=d$.}
{a message $m$ such that, given a digest $d$, $h(m)=d$.}
{a message $m$ such that, given $m_0$, $h(m)=h(m_0)$.}
{two different messages $m_1$, $m_2$ such that $h(m_1)=h(m_2)$."}

\question[1]{Tick the $ \textbf{true} $ assertion.}
{In an universal forgery the adversary has stronger objectives than in an existential forgery.}
{In a MAC forgery under known message attack the adversary is able to request the authentication of several messages.}
{In an existential forgery the adversary is able to forge a valid MAC for an arbitrary message.}
{A MAC provides authentication, integrity and confidentiality."}

\question[4]{We assume that the factorization of $n$ is known. The complexity to find a square root modulo $n$, where $\ell$ is the bitlength of $n$ is ($ \textbf{choose the most accurate answer} $):}
{$\mathcal{O}(\ell^4)$.}
{$\mathcal{O}(\log \ell)$.}
{$\mathcal{O}(\ell)$.}
{$\mathcal{O}(\ell^3)$."}

\question[2]{What adversarial model does not make sense for a hash function?}
{collision attack.}
{universal forgery.}
{second preimage attack.}
{first preimage attack."}

\question[3]{Tick the $ \textbf{incorrect} $ assertion. Given $p$ prime:}
{the cardinality of $\mathbb{Z}_p^*$ is $p-1$.}
{the group of quadratic residues modulo $p$ is of order $\frac{p-1}{2}$.}
{for any $x \in \mathbb{Z}_p$, $x^{p-1} \equiv 1 \pmod{p}$.}
{$\mathbb{Z}_p^*$ is a cyclic group."}

\question[4]{Tick the $ \textbf{incorrect} $ statement. When $x\rightarrow+\infty$ \ldots}
{$x^3 + 2x + 5 = \mathcal{O}(x^3)$.}
{$\frac{1}{x^2} = \mathcal{O}(\frac{1}{x})$.}
{$2^{\frac{x}{\log x}} = \mathcal{O}(2^x)$.}
{$n^x = \mathcal{O}(x^n)$ for any constant $n>1$."}

\question[3]{Tick the $ \textbf{incorrect} $ assumption. A language $L$ is in NP if\dots}
{$x \in L$ can be decided in polynomial time.}
{$x \in L$ can be decided in polynomial time given a witness $w$.}
{$L$ is NP-hard.}
{$L$ (Turing-)reduces to a language $L_2$ with $L_2$ in $P$, i.e., if there is a polynomial deterministic Turing machine which recognizes $L$ when plugged to an oracle recognizing $L_2$."}

\question[4]{Tick the $ \textbf{incorrect} $ assertion. Let $H:\left\{ 0,1 \right\}^*\rightarrow\left\{ 0,1 \right\}^n$ be a hash function.}
{We can use $H$ to design a commitment scheme.}
{We can use $H$ to design a key derivation function.}
{Finding $x,y\in\left\{ 0,1 \right\}^*$ such that $x\neq y$ and $h(x) = h(y)$ can be done in $O(2^{n/2})$ time.}
{Given $x\in\left\{ 0,1 \right\}^*$, finding a $y \in \left\{ 0,1 \right\}^*$ such that $x\neq y$ and $h(x) = h(y)$ can be done in $O(2^{n/2})$ time."}

\question[2]{Tick the most accurate answer. Generating an $\ell$-bit prime number can be done in}
{$O(\ell^2)$ time.}
{$O(\ell^4)$ time.}
{$O(\ell^3)$ time.}
{$O(\ell)$ time."}

\question[3]{Which of the following is $ \textbf{not} $ a finite field.}
{The set of integers modulo $p$, where $p$ is a prime number.}
{$Z_2[X]/(X^2+X+1)$.}
{The elliptic curve $E_{2,1}$ over $GF(7)$.}
{$GF(p^n)$, where $p$ is a prime number and $n$ a positive integer."}

\question[1]{Suppose that you can prove the security of your digital signature scheme against the following attacks. In which case is your scheme going to be the $ \textbf{most} $ secure?}
{Existential forgery under chosen message attack.}
{Universal forgery under chosen message attack.}
{Existential forgery under known message attack.}
{Universal forgery under known message attack."}

\question[4]{Suppose that you can prove the security of your symmetric encryption scheme against the following attacks. In which case is your scheme going to be the $ \textbf{most} $ secure?}
{Key recovery under known plaintext attack.}
{Key recovery under chosen ciphertext attack.}
{Decryption under known plaintext attack.}
{Decryption under chosen ciphertext attack."}

\question[2]{Tick the $ \textbf{correct} $ assertion.}
{$x \in \mathbb{Z}_n$ iff $\gcd(x,n)=1$.}
{In a group, the order of an element divides the group exponent.}
{All finite rings have cardinality $p^n$, where $p$ is a prime and $n \in \mathbb{N}$.}
{We always have a ring isomorphism between $\mathbb{Z}_{mn}$ and $\mathbb{Z}_n \times \mathbb{Z}_m$."}

\question[1]{The exponent of the group $\mathbb{Z}_9^*$ is}
{6.}
{9.}
{8.}
{3."}

\question[2]{Let $H$ be a hash function $H: \{0,1\}^* \rightarrow \{0,1\}^n$. In order to provide a complexity of at least $2^{128}$ against generic attacks, we should pick $n$ to be}
{128.}
{256.}
{196.}
{80."}

\question[4]{What adversarial model does not make sense for a message authentication code (MAC)?}
{key recovery.}
{universal forgery.}
{existential forgery.}
{decryption."}

\question[4]{Tick the \textbf{incorrect} assertion. For two independent random variables $X,Y$ with the range in real numbers, we have that \ldots}
{$\phantom{\mathrm{ar}}\mathrm{E}[X+Y]=\phantom{\mathrm{ar}}\mathrm{E}[X]+\mathrm{E}[Y]$}
{$\phantom{\mathrm{ar}}\mathrm{E}[X \mkern4.5mu \cdot \mkern4.5mu Y]=\phantom{\mathrm{ar}}\mathrm{E}[X] \mkern4.5mu \cdot \mkern4.5mu \mathrm{E}[Y]$}
{$\mathrm{Var}[X+Y]=\mathrm{Var}[X]+\mathrm{Var}[Y]$}
{$\mathrm{Var}[X \mkern4.5mu \cdot \mkern4.5mu Y]=\mathrm{Var}[X] \mkern4.5mu \cdot \mkern4.5mu \mathrm{Var}[Y]$"}

\question[1]{For which of the following primitives, there is no notion of security against a distinguisher?}
{A digital signature scheme.}
{A public key encryption scheme.}
{A symmetric encryption scheme.}
{A pseudo-random function (PRF)."}

\question[3]{Which of the following primitives \textbf{cannot} be instantiated with a cryptographic hash function?}
{A pseudo-random number generator.}
{A commitment scheme.}
{A public key encryption scheme.}
{A key-derivation function."}

\question[2]{What is $\lambda (91)$, i.e. the exponent of $\mathbb{Z}_{91}^*$?}
{90}
{12}
{72}
{6"}

\question[4]{Tick the \textbf{correct} assertion.}
{For any prime $p$ and a positive integer $\alpha$, $\mathbb{Z}_{p^\alpha}$ is a finite filed.}
{For a prime $p$ and a positive integer $n$, we have $|GF(p^n)^*|=p^{n-1}(p-1)$.}
{$\mathbb{Z}_{11}^*$ contains elements of order $11$.}
{There is a ring isomorphism between $\mathbb{Z}_m \times \mathbb{Z}_n$ and $\mathbb{Z}_{mn}$ if and only if $\gcd(m,n)=1$."}

