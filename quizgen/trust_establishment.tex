\question[2]{The names of Needham and Schroeder refers to $ \ldots $}%
{an authentication protocol involving only a client and a server.}%
{an authentication protocol involving a client, an authentication server, and a server.}%
{a theorem about the security of Kerberos.}%
{the inventors of GSM.}%

\question[1]{In Kerberos, a client that would like to authenticate to a server (S) has to interact successively with three entities that are: the server (S), the Authentication Server (AS), the Ticket Granting Server (TGS). In which order does the client have to interact with the entities?}%
{AS-TGS-S} %
{S-TGS-AS-S}% 
{AS-S-TGS}%
{TGS-S-AS-S}%

\question[4]{In a password-based challenge/response client authentication protocol, $ \ldots $}%
{the challenge is sent by the client.}
{the challenge is sent after the response.}
{the password is sent in clear.}
{the server must keep the password and protect the database.}

\question[1]{In Kerberos, a client that would like to authenticate to a server (S) has to interact successively with three entities that are: the server (S), the Authentication Server (AS), the Ticket Granting Server (TGS). In which order does the client have to interact with the entities?}
{AS-TGS-S} 
{S-TGS-AS-S} 
{TGS-AS-S}
{TGS-S-AS-S}

\question[4]{In a MAC-based challenge-response authentication protocol $ \ldots $}
{the challenge is sent by the client.}
{the challenge is sent after the response.}
{the password is sent in clear.}
{the server must keep the password and protect the database.}



